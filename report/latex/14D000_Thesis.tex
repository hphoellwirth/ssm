\documentclass[11pt, oneside]{scrreprt}   	% use "amsart" instead of "article" for AMSLaTeX format
\usepackage{geometry}                		% See geometry.pdf to learn the layout options. There are lots.
\geometry{letterpaper}                   		% ... or a4paper or a5paper or ... 
%\geometry{landscape}                		% Activate for rotated page geometry
%\usepackage[parfill]{parskip}    		% Activate to begin paragraphs with an empty line rather than an indent
\usepackage{graphicx}				% Use pdf, png, jpg, or eps§ with pdflatex; use eps in DVI mode
								% TeX will automatically convert eps --> pdf in pdflatex		
\usepackage{amssymb}
\usepackage{amsmath,amsfonts,amsthm} % Math packages
\usepackage{bm}
\usepackage{graphicx}
\graphicspath{ {images/} }


%%%%%%%%
%    Cover     %
%%%%%%%%
\title{\textit{}\\\textit{}\\State Space Models}
\author{Hans-Peter H{\"o}llwirth}
\publishers{Master Project Report \\ 
Barcelona Graduate School of Economics \\ Master Degree in Data Science \\ 2017}
\date{}

\begin{document}
\maketitle
%\afterpage{\blankpage}

%%%%%%%%%%%%%
%    Table of Contents   %
%%%%%%%%%%%%%
\newpage
\tableofcontents
\newpage

%%%%%%%%%%
%    Introduction   %
%%%%%%%%%%
\chapter{Introduction}
\label{chp:introduction}



%%%%%%%%
%    Models   %
%%%%%%%%
\chapter{State Space Models}
\label{chp:models}


%%%%  Local Level Model  %%%%
\section{Local Level Model}

\begin{center}
\begin{tabular}{ r r l }
  observation: & $y_t = x_t + \epsilon_t$, & $\epsilon_t \sim N(0,\sigma_{\epsilon}^2)$ \\
  state: & $x_{t+1} = x_t + \eta_t$, & $\eta_t \sim N(0,\sigma_{\eta}^2)$ \\
\end{tabular}
\end{center}


%%%%  Trivariate Local Level Model  %%%%
\section{Trivariate Local Level Model}

\begin{center}
\begin{tabular}{ r r l }
  observation: & $y_t = \begin{bmatrix}y_{1t}\\ y_{2t} \\ y_{3t} \end{bmatrix} = x_t + \epsilon_t$, & $\epsilon_t \sim N(\textbf{0}, \begin{bmatrix}\sigma_{\epsilon 1}^2 \\ \sigma_{\epsilon 2}^2 \\ \sigma_{\epsilon 3}^2 \end{bmatrix} I)$ \\
  state: & $x_{t+1} = x_t + \eta_t$, & $\eta_t \sim N(\textbf{0}, \Sigma_{\eta})$ \\
\end{tabular}
\end{center}



%%%%  Hierarchical Dynamic Poisson Model   %%%%
\section{Hierarchical Dynamic Poisson Model}
Let $t$ denote the day and $i$ be the intraday index.

\begin{center}
\begin{tabular}{ r r l }
  %observation: & $y_{ti} = x_{ti} + \epsilon_t$, & $\epsilon_t \sim N(0,\sigma_{\epsilon}^2)$ \\
  state: & $x_{ti} = \text{Poisson}(\lambda_{ti})$\\
\end{tabular}
\end{center}

where the parameter
$$
\log \lambda_{ti} = \log \lambda_t^{(D)} + \log \lambda_i^{(P)} + \log \lambda_{ti}^{(I)}  
$$
consists of a daily, a periodic, and an intra-daily component:
\begin{center}
\begin{tabular}{ r l l }
  daily component: & $\log \lambda_t^{(D)} = \phi_0^{(D)} + \phi_1^{(D)} \log \lambda_{t-1}^{(D)}  + u_t^{(D)}$ & (AR(1)) \\
  periodic component: & $\log \lambda_i^{(P)} = \delta_i$ & (B-spline) \\
  intra-daily component: & $\log \lambda_{ti}^{(I)} = \phi_1^{(I)} \log \lambda_{ti-1}^{(I)}  + u_{ti}^{(I)}$ & (AR(1)) \\
\end{tabular}
\end{center}


%%%%%%%
%  Filtering %
%%%%%%%
\chapter{Filtering}
\label{chp:filtering}
The object of filtering is to update our knowledge of the system each time a new observation $y_t$ is brought in.

%%%%  Kalman Filter   %%%%
\section{Kalman Filter}

%\begin{align} 
%\begin{split}
%|m-M| &= |\mathbb{E}(X)-M|\\
%\end{split}					
%\end{align} 


%%%%%%%%%%
%    Conclusion   %
%%%%%%%%%%
\chapter{Conclusion}
\label{chp:conclusion}
by Etessami et al.\cite{etessami2014_2}

%%%%%%%%%%%
%    Bibliography          %
%%%%%%%%%%%
%\afterpage{\blankpage}
\bibliography{references}
\bibliographystyle{plain}
\end{document}  















